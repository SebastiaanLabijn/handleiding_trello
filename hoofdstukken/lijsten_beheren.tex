\chapter{Lijsten beheren}\label{chapter:lijsten_beheren}

Aangezien er tijdens de projecten volgens het Agile principe wordt gewerkt is het nodig om de verschillende kaarten/taken te organiseren en deze onder te brengen in hun juiste status gedurende de iteratie. De status van een kaart/taak wordt bepaald door de lijst (kolom op het bord) waarin deze zich bevindt.

\noindent
\\Maak, per bord, lijsten aan die fasen voorstellen waarin elke taak zich zal bevinden. De naamgeving in het vet voor de / is voor programmeurs, de naamgeving in het cursief na de / is voor de systeembeheerders:
\begin{itemize}
	\item \textbf{Product Backlog} / \textit{Backlog}: Hierin komen alle kaarten die tijdens het volledige project nog moeten worden afgewerkt. Deze kolom bevat dus de resterende kaarten voor het project zelf en bevat kaarten over de borden heen. De kaarten zelf bevatten nog \textbf{geen} inschattingen;
	\item \textbf{Sprint Backlog} / \textit{Ready}: Dit is een lijst die alle kaarten bevat die op dit bord zullen worden uitgevoerd. Deze kaart wordt verplaatst vanuit de lijst ``Product backlog''. Tijdens het verplaatsen wordt ook de inschatting van de teamleden aan de kaart toegevoegd;
	\item \textbf{Doing} / \textit{Working}: Dit zijn kaarten waar het team momenteel effectief aan werkt;
	\item \textbf{Testing} / \textit{Testing}: Deze lijst bevat de kaarten die uitgevoerd zijn maar nog getest dienen te worden;
	\item \textbf{Done} / \textit{Ready for Review}: Dit zijn kaarten die volledig afgewerkt zijn.
\end{itemize}
Het is ook mogelijk om zelf nog extra lijsten toe te voegen. Indien een nieuw bord gemaakt wordt voor de volgende sprint wordt steeds de lijst ``product backlog'' integraal van het vorige bord overgenomen met behulp van ``Lijst kopi\"eren'' (Programmeren).

\section{Lijst aanmaken}
Om een lijst aan te maken onderneem je volgende stappen:
\begin{enumerate}[nolistsep]
	\item Selecteer het gewenste bord;
	\item Kies de optie ``Voeg een lijst toe...'';
	\item Voer de gewenste naam in;
	\item Kies ``Bewaar''.
\end{enumerate}

\section{Lijst wijzigen}
Er zijn verschillende wijzigingen die je aan een lijst kan aanbrengen.
\begin{itemize}
	\item Naam wijzigen
	\begin{enumerate}
		\item Klik op de naam van de lijst die je wenst te wijzigen;
		\item Voer de nieuwe naam in;
		\item Bevestig de wijzigingen.
	\end{enumerate}
	\item Plaats van de lijst wijzigen	
	\begin{enumerate}
		\item Selecteer de gewenste lijst;
		\item Sleep deze naar de gewenste plaats.
	\end{enumerate}
	\item Abonneren (Updates ontvangen indien lijst wijzigt)
	\begin{enumerate}
		\item Selecteer ``...'' op de gewenste lijst;
		\item Selecteer ``Abonneren''.
	\end{enumerate}
\end{itemize}
\noindent
\\Indien je een abonnement wil opzeggen voor een bepaalde lijst herhaal je gewoon de stappen voor ``Abonneren''.

\section{Lijst verplaatsen}

Het is ook mogelijk om een lijst en al zijn bijhorende kaarten te verplaatsen (zelfs naar een ander bord). Om dit te realiseren dienen volgende stappen uitgevoerd te worden:
\begin{enumerate}[nolistsep]
	\item Selecteer ``...'' op de gewenste lijst;
	\item Kies ``Verplaats lijst ..";
	\item Kies het gewenste ``bord'' en ``positie" (= kolomnummer) binnen dat bord;
	\item Bevestig met ``verplaats".
\end{enumerate}

\section{Lijst kopi\"eren}

Een lijst en al zijn bijhorende kaarten kunnen \textbf{binnen eenzelfde bord} op volgende manier gekopieerd worden:
\begin{enumerate}[nolistsep]
	\item Selecteer ``...'' op de gewenste lijst;
	\item Kies ``Kopieer lijst ..";
	\item Voer de gewenste ``naam'' in;
	\item Bevestig met ``maak een lijst aan".
\end{enumerate}

\section{Lijst verwijderen}

Om een lijst te verwijderen van een Trellobord dient de optie ``lijst archiveren'' geactiveerd te worden. Deze is te bereiken via de optie ``...'' bij de gewenste lijst en dan ``Archiveer lijst''.\\
 \textbf{AANDACHT:} Trello vraagt niet om een bevestiging, dus de lijst en al zijn bijhorende kaarten zijn onmiddellijk verwijderd!